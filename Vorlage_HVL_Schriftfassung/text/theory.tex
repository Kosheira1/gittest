\chapter{Theory}
\section{Debye model}
The polarization effects in a dielectric material can be characterized by the relative permittivity and the dielectric loss factor. 
For linear materials the polarization can be described by a network model. The Debye-ansatz assumes that the rate of change $ dP_i/dt$ is of the remaining difference between the polarization $P_i(t)$ and the polarization at infinity $P_i(\infty$) is proportional. This results in an exponential decay of the polarization function that towards $P_i(\infty)$. This exponential decay can be modeled with a time constant of a resistor and a capacitor in series ($\tau=R_i \cdot C_i$). 

\subsection{Derivation of maximum tan($\delta$)}
The aim of this section is to derive the formula for the maximum tan($\delta$) of the Debye model in order to adjust its value to a disired one.
\begin{equation}
\epsilon_{eff}^* (\omega) = \frac{C^*(\omega)}{C_0}
\end{equation}

\begin{equation}
\epsilon^* = \epsilon_r-j \cdot \epsilon_r''
\end{equation}


\begin{equation}
\epsilon'_r = \epsilon_{\infty} + \frac{\epsilon_{stat}-\epsilon_{\infty}}{1+(\omega \cdot \tau )^2}
\end{equation}

\begin{equation}
\epsilon''_r = \omega \cdot \tau \cdot \frac{\epsilon_{stat}-\epsilon_{\infty}}{1+(\omega \cdot \tau )^2}
\end{equation}

The loss tangents are given by:
\begin{equation}
tan (\delta) = \frac{\kappa + \omega \cdot \epsilon_0 \cdot \epsilon _r ''}{\omega \cdot \epsilon_0 \cdot \epsilon _r '}
\end{equation}

\begin{equation}
tan (\delta_{L}) = \frac{\kappa}{\omega \cdot \epsilon_0 \cdot \epsilon_r'} \newline
\end{equation}

\begin{equation}
tan (\delta_{pol}) = \frac {\epsilon_r'' } {\epsilon_r'}
\end{equation}


We are measuring in the frequency domain from $10^{-6}$ Hz to $10^{11}$ Hz 

\begin{equation}
\epsilon^*(\omega) = \frac{1}{j \omega  Z^*(\omega) C_0}
\end{equation}

\begin{equation}
Z_0(\omega)=[\frac{1}{R_\infty}+\frac{1}{R_i+\frac{1}{j \omega C_i}}+j \omega C_0]^{-1} = [\frac{1}{R_\infty}+\frac{j \omega C_i}{j\omega R_i  C_i+1}+j \omega C_0]^{-1}
\end{equation}

\begin{equation}
\epsilon^*(\omega)= \frac{[\frac{1}{R_\infty}+\frac{j \omega C_i}{j\omega C_i R_i  +1}+j \omega C_0]}{j \omega C_0} = \frac{1}{j \omega C_0 R_\infty}+ \frac{C_i/C_0}{j\omega C_i R_i  +1}+1
\end{equation}

We split this term up into real and impaginary part 

\begin{equation}
\epsilon_r' = 1+ \frac{C_i/C_0}{\omega^2 C_i^2 R_i^2 +1}
\end{equation}

\begin{equation}
\epsilon_r'' = -j \left(\frac{1}{\omega C_0 R_\infty}+\frac{\omega C_i^2 R_i / C_0}{\omega^2 C_i^2 R_i^2 +1} \right)
\end{equation}

\begin{equation}
tan(\delta) = tan(\delta_L) + tan( \delta_Pol) = \frac{\omega^2 \tau^2+1}{\omega C_0 R_\infty (2+ \omega^2 \tau^2)}=\frac{\omega \tau \Delta \epsilon}{\epsilon_{stat} + \omega^2 \tau^2}
\end{equation}

\begin{equation}
\frac{\partial tan(\delta)}{ \partial \omega} = \frac{[\omega C_0 R_\infty (2+\omega^2 \tau^2)]\cdot 2 \omega \tau^2 - (\omega^2 \tau^2 +1) [C_0 R_\infty (2+3 \omega^2 \tau^2)  }{[\omega C_0 R_\infty (2+\omega^2 \tau^2)]^2}+ \frac{\tau \Delta \epsilon [\epsilon_{stat} + \omega^2 \tau^2] - 2 \omega \tau^2 [\omega \tau \Delta \epsilon]}{[\epsilon_{stat} +\omega^2 \tau^2]}
\end{equation}

Um das Maximum von tan($\delta_{pol}$) zu finden wird der zweite Term in der Ableitung 0 gesetzt.

\begin{equation}
\tau \Delta \epsilon [\epsilon_{stat} + \omega^2 \tau^2] -2\omega \tau^2 [\omega \tau \Delta \epsilon] = 0
\end{equation}
\begin{equation}
\omega^2 (\tau^3 \Delta \epsilon -2 \tau^3 \Delta \epsilon) = - \tau \Delta \epsilon \cdot \epsilon_{stat}
\end{equation}
\begin{equation}
\omega^2 = \frac{\tau \Delta \epsilon \epsilon_{stat}}{\tau^3 \Delta \epsilon}
\end{equation}
\begin{equation}
\omega = \sqrt{\frac{\epsilon_{stat}}{\tau^2}}
\end{equation}
\begin{equation}
\tan(\delta_L)_{max} = \frac{\epsilon_{stat} \Delta\epsilon}{2\epsilon_{stat}} = \frac{1}{2} \frac{\Delta \epsilon}{\sqrt{\epsilon_{stat}}}
\end{equation}


\section{Fourier Coefficients of trapezoidal pulse  train }
\begin{equation}
 U_n = \frac{2 U_0}{j \omega_n T} [e^{\frac{j \omega_n \tau}{2}} sinc{\frac { \omega_n \tau_r }{2}} -e^{\frac{-j \omega_n \tau}{2}} sinc{\frac{ \omega_n \tau_f}{2}}]
\end{equation}
 
System Analysis: 
 
 
\begin{equation}
 \tau_r = 0.5e^{-6 }
  \end{equation}
  \begin{equation}
 \tau_f = 0.1e^{-6}
  \end{equation}
 \begin{equation}
\tau= 1e^{-6}
 \end{equation}
 
Slopes of the curves: \newline
-1 on loglog plot after $\frac{2}{\tau}$ \newline
-2 on loglog plot after $\frac{2}{\tau_r}$


\section{Clamping capability of Current Transformers }
