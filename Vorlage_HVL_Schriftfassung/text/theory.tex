\chapter{Theory}

\section{Model of electron swarm}

We assume that at an arbitrary position $z = d_0$ an initial temporal electron number distribution can be observed,
\begin{equation}
n_e(d_0, t) = \frac{n_0}{ \sqrt{2 \pi \sigma_0^2} } \, \exp{\frac{(t - t_0)^2}{2 \sigma_0^2} }  \;,
\label{eq.swarmstart}
\end{equation}
with the number of initial electrons $n_0$ and initial temporal broadening $\sigma_0$. After a travel time $T$, when this swarm passes an observer at position $d = d_0 + \mathbf{w} T$, the temporal distribution is
\begin{equation}
n_e(d, t) = \frac{n_0 \, \exp{\nu_\mathrm{eff} (t-t_0)}}{ \sqrt{2 \pi \sigma^2(t)} } \, \exp{\frac{(t - t_0 - T)^2}{2 \sigma^2(t)} }
\label{eq.swarmend}
\end{equation}
The total number of electrons was changed at an effective rate $\nu_\mathrm{eff}$. This effective rate accounts for electron attachment, ionization, electron detachment and photoionization by secondary photons. However $\nu_\mathrm{eff}$ is not affected by ionic or metastable species accumulated in the gas, because the interval between two subsequent measurements is longer than their residence time. The swarm temporal broadening $\sigma(t)$ was increased by the diffusion time constant $\tau_\mathrm{D}$:
\begin{equation}
\sigma^2(t) = \sigma_0^2 + \tau_\mathrm{D} (t-t_0)
\label{eq.diffusiontime}
\end{equation}

\subsection{More examples}

\begin{equation*}
    \cos (2\theta) = \cos^2 \theta - \sin^2 \theta
\end{equation*}

\begin{equation}
\left. \left( \frac{E}{N} \right) \right|_\mathrm{crit}
\end{equation}

\begin{equation}
   \mathrm{d}f = \sum_{i=1}^{n} \frac{\partial f}{\partial x_i} \mathrm{d}x_i 
\end{equation}


\begin{equation}
   L = 35\,\mathrm{nH}\,, \hskip 5mm  C = 50\,\mathrm{\mu F}\,, \hskip 5mm U_\mathrm{arc}=10\ldots50\,\mathrm{V}\,, \hskip 5mm P=10'000\,\mathrm{W}\,. 
\end{equation}



