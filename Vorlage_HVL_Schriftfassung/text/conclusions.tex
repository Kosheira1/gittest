\chapter{Conclusions}
\section{Estimation of current $\epsilon$}
The created lookup-table allows an estimation of the the electrode distance d.  As shown in the chapters before, other factors than the air gap can be neglected. The calculated correction factor allows to estimate the $\epsilon$ based on $\epsilon_eff$ improves the guess of $\epsilon$. Thus, within the range of the created look-up table the aims of this part of the semester project were reached. 

It has to be noted that this method is based on the assumption that the initial $\epsilon$ is always the same. This work did not investigate to what extend this assumption is tenable. 

\section{Statement on the suitability of the current transformer for dielectric spectroscopy}

The number of measurements for the polarization losses is quite low, therefore, it would be useful to increase the number of input frequencies in order to validate whether the measured values approximately possess the shape of the function of tan($\delta$) over a larger area and how large the deviation from the theoretical value is. This necessitates an automation of the gain adaptation for the amplifier of the butter-worth filter in order to use the resolution scope of the ADC ($\pm 10V$)  effectively. 
The band of the confidence interval for conduction losses is approx. 48 \% of the midpoint's value. This indicates that the noise level is too large to make a serious guess of the mean of the real tan($\delta)$.





