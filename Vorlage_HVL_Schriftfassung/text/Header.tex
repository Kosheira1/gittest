%****Header*****

\documentclass[%draft,
fleqn,											%richtet die Formel links aus, Einzug wird unten definiert
12pt,a4paper,
%twoside,openright,					%fuer einseitigen Druck diese beiden Optionen auskommentieren und unten ensprechende Kopfzeilenoption waehlen (bis ca 120 Seiten)
bibliography=totoc,listof=totoc,BCOR=5mm,
ngerman
%english
]{scrreprt}

%%%%%%%%%%%%%%%%%%%%%%%%%%%%%%%%%%%%%%%%%%%
%Textliches und sprachliches
%%%%%%%%%%%%%%%%%%%%%%%%%%%%%%%%%%%%%%%%%%%
\usepackage[T1]{fontenc}		% die beiden Packete geh�ren zusammen, um Umlaute und �hnliche deutsche "Sonderzeichen"
\usepackage[latin1]{inputenc} %direkt eingeben zu k�nnen
\usepackage[english]{babel}%kommt es zu einer Uebersetzung der sprachspezifischen Befehle im Dokument (�Inhaltsverzeichnis� statt �table of contents�, �Kapitel� anstelle von �chapter� usw.) und fuer korrekte Trennung wichtig
\usepackage{upgreek} 				%erlaubt die Eingabe von aufrechten griechischen Kleinbuchstaben
\pagestyle{headings} 				%schaltet die Kopfzeile ein. Die Option {headings} erzeugt die Nummer und den Namen des Kapitels
\usepackage{setspace} 			%enth�lt u.a. den Befehl \onehalfspacing, der daf�r sorgt, dass 1 1/2 Zeilen Abstand gelassen wird zwischen zwei Zeilen
\usepackage{rotating}
\usepackage{pdflscape}
\usepackage[]{nomencl}  % Nomenclature Symbolverzeichnis
\usepackage[euler]{textgreek}
%%%%%%%%%%%%%%%%%%%%%%%%%%%%%%%%%%%%%%%%%%%
%Verzeichnisse
%%%%%%%%%%%%%%%%%%%%%%%%%%%%%%%%%%%%%%%%%%%
\setcounter{secnumdepth}{2} %bestimmt, in welcher Tiefe Ueberschriften noch nummeriert werden. 5 Steht dafuer, dass alle Ueberschriften nummeriert werden
\setcounter{tocdepth}{1} 		%Bestimmt die Nummerierungstiefe im Inhaltsverzeichnis
\usepackage{cite}						%braucht man f�rs Literaturverzeichnis
\usepackage{bibgerm} 				%F�r die Literaturverzeichniserstellung auf deutsch
\usepackage{epstopdf}			% F�r eps Bilder
%%%%%%%%%%%%%%%%%%%%%%%%%%%%%%%%%%%%%%%%%%%
%Mathematisches
%%%%%%%%%%%%%%%%%%%%%%%%%%%%%%%%%%%%%%%%%%%
\usepackage{amssymb,amsfonts,amsmath} %Eingabehilfen f�r Formeln
\usepackage{siunitx} 				%erlaubt die Eingabe von Einheiten, so dass die Einheit und der Wert schoen zusammenstehen und nicht getrennt werden beim Zeilenumbruch.

%%%%%%%%%%%%%%%%%%%%%%%%%%%%%%%%%%%%%%%%%%%
%Bildtechnisches
%%%%%%%%%%%%%%%%%%%%%%%%%%%%%%%%%%%%%%%%%%%
\usepackage{graphicx} 			%um Bilder einzubinden
\usepackage[table]{xcolor}
\usepackage{float} 		%erlaubt mehr float Objekte (z.B. Bilder oder Tabellen) auf einer Seite
%\usepackage{subfig}				%wenn mehrere Bilder unter einer Caption zusammengefasst werden sollen
\usepackage[center,small,bf]{caption} %gibt die Optionen f�r das Verhalten der Bildunterschriften (caption) an, muss nach Paketen wie subfig, float, rotating eingebunden werden

%%%%%%%%%%%%%%%%%%%%%%%%%%%%%%%%%%%%%%%%%%%
%Fuss- und Kopfzeilendefinition
%%%%%%%%%%%%%%%%%%%%%%%%%%%%%%%%%%%%%%%%%%%
\usepackage{fancyhdr}
\fancyhead{}
\fancyfoot{}
\renewcommand{\headrulewidth}{0.5pt}

%%% zweiseitiges Seitenlayout
%\fancyhead[EL, OR]{ \thepage}
%\fancyhead[EC]{\textsl{\leftmark}}
%\fancyhead[OC]{\textsl{\rightmark}}
%%% Ende: zweiseitiges Seitenlayout

%%% einseitiges Seitenlayout
\fancyhead[R]{ \thepage}
\fancyhead[C]{\textsl{\leftmark}}
%%% Ende: einseitiges Seitenlayout

\pagestyle{fancy}

\renewcommand{\chaptermark}[1]{%
\markboth{\thechapter\ #1}{}}

\renewcommand{\sectionmark}[1]{%
\markright{\thesection\ #1}}

%------------------------------------------------------------------------------
%Bis hierhin darf der Header nur in Ruecksprache mit dem Betreuer geaendert werden.
%------------------------------------------------------------------------------

%%%%%%%%%%%%%%%%%%%%%%%%%%%%%%%%%%%%%%%%%%%
%Definitionen f�r siunitx-Paket
%wird nur in deutscher Version benoetigt
%%%%%%%%%%%%%%%%%%%%%%%%%%%%%%%%%%%%%%%%%%%
\sisetup{locale=DE} 			%macht, dass statt Punkt ein Komma als Dezimaltrenner verwendet wird
\sisetup{%
  list-final-separator = { \translate{and} },
  range-phrase = { \translate{to (numerical range)} }
} 												%macht dass in Liste und Range die Woerter richtig uebersetzt werden

%%%%%%%%%%%%%%%%%%%%%%%%%%%%%%%%%%%%%%%%%%%
%Einbinden mehrseitiger PDFs oder Ausschnitte daraus
%%%%%%%%%%%%%%%%%%%%%%%%%%%%%%%%%%%%%%%%%%%
%\usepackage{calc}
%\usepackage{eso-pic}
%\usepackage{everyshi}
%\usepackage{pdfpages}

%%%%%%%%%%%%%%%%%%%%%%%%%%%%%%%%%%%%%%%%%%%
%Fuer das Erstellen von Bildern aus Matlab mit \psfragfig
%%%%%%%%%%%%%%%%%%%%%%%%%%%%%%%%%%%%%%%%%%%
%Fuer die Nutzung von \psfragfig muessen die Bilder mit Hilfe des figures_fuer_latex.m erzeugt werden -- es muss dem Complier noch das Argument "-shell-escape" uebergeben werden (bei Texniccenter Alt+F7)
%\usepackage{pstool}
\usepackage[crop=pdfcrop]{pstool} %verschmaelert die Raender der Abbildung
%\usepackage[cleanup={}]{pstool} 	 %log-Datei bleibt erhalten: zum Debuggen

%%%%%%%%%%%%%%%%%%%%%%%%%%%%%%%%%%%%%%%%%%%
%Fuer den Import von Bildern aus Inkscape
%%%%%%%%%%%%%%%%%%%%%%%%%%%%%%%%%%%%%%%%%%%
%\graphicspath{{figures/}}
%\newcommand{\executeiffilenewer}[3]{%
%\ifnum\pdfstrcmp{\pdffilemoddate{#1}}%
%{\pdffilemoddate{#2}}>0%
%{\immediate\write18{#3}}\fi%
%}
%\newcommand{\includesvg}[1]{%
%\executeiffilenewer{#1.svg}{#1.pdf}%
%{inkscape -z -D --file=#1.svg %
%--export-pdf=#1.pdf --export-latex}%
%\input{#1.pdf_tex}%
%}

%%%%%%%%%%%%%%%%%%%%%%%%%%%%%%%%%%%%%%%%%%%
%Fuer TpX
%Mit TpX koennen unter Window recht einfach Diagramme erstellt werden
%%%%%%%%%%%%%%%%%%%%%%%%%%%%%%%%%%%%%%%%%%%
%\usepackage{color,ifpdf,graphicx}
%\ifpdf %if using PDFTeX in PDF mode
%  \DeclareGraphicsExtensions{.pdf,.png,.mps}
%\else %if using TeX or PDFTeX in TeX mode
%  \DeclareGraphicsExtensions{.eps,.bmp}
%  \DeclareGraphicsRule{.emf}{bmp}{}{}% declare EMF filename extension
%  \DeclareGraphicsRule{.png}{bmp}{}{}% declare PNG filename extension
%  \usepackage{pstricks}%variant: \usepackage{pst-all}
%\fi
%\usepackage{pgf,epic,bez123,floatflt,wrapfig}

%%%%%%%%%%%%%%%%%%%%%%%%%%%%%%%%%%%%%%%%%%%
%Neu-/Umdefinitionen
%%%%%%%%%%%%%%%%%%%%%%%%%%%%%%%%%%%%%%%%%%%
\usepackage[pdfborder={0 0 0}, breaklinks=true]{hyperref} %muss als letztes Paket eingebunden werden, verlinkt Referenzen, so dass im pdf direkt zu der Stelle gesprungen wird

%%%%%%%%%%%%%%%%%%%%%%%%%%%%%%%%%%%%%%%%%%%
%Neu-/Umdefinitionen
%%%%%%%%%%%%%%%%%%%%%%%%%%%%%%%%%%%%%%%%%%%

\newcommand{\sfs}{SF$_6$} 