\chapter{Discussion}

\section{Measures to improve the integrator}
The results indicate that the integrator has much noise at low frequencies. One might think that this does not influence the results as frequencies below 1kHz are not monitored, but this due to the spectral leakage of the FFT not the case.  A high-pass filter at the output of the integrator could reduce these parasitic frequency components. This might reduce the measured swing at the output of the integrator and thus improve the results of the dielectric spectroscopy. 
Moreover, a more detailed analysis of the origin of these low-frequency components is necessary. It has already been shown that the external voltage source is not responsible for them, thus the integrator has to be investigated part by part in order to find the origin of this noise. 

Another measure to reduce the offset actively is a feedback loop. A PI-controller in the feedback loop at the output can make sure that there an offset at the input is erased as described in \cite{thomas}. 

\section{Measures to improve the current transformer}
As shown in the chapter \ref{chp.results} the measurements with the current transformer deteriorates the confidence interval of the measured $tan(\delta)$ by orders. This deterioration is reduced by the integrator, but it is still higher than with a shunt resistor. 
Furthermore, a more detailed analysis of the correlation between the NF of the current transformer (with and without integrator) for different frequencies and the confidence interval of the measured $tan(\delta)$ is necessary. This would provide detailed information what impact the quality of the current transformer has on the precision of the dielectric spectroscopy. 
The error class of the used Pearson CT 2877 is $+1\/-0\%$, which includes current and phase errors as well as well as effects of harmonics in the primary current according to IEC 600-44-1. The phase shift errors of the CT 2877 are given by its Application Note \footnote{ http://www.pearsonelectronics.com/pdf/Application_20Notes.pdf} as below 6 degrees for frequencies at least one decade above the cutoff frequency. As 1 kHz is not one decade above the the corner frequency of 300 MHz. 