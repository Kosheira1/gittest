\chapter{Discussion}

\section{Simulations}
Given the fact that the resulting effective dielectric permittivity, which
is mostly based on the simulation results has 
a reasonable value as well as the general dependency of the capacitance on electrode distance being consistent
with the approximation of two spheres once we factor in the third electrode in the geometry leads one 
to believe that the simulations are reliable. 
\newline
The only thing that initially seemed not credible was the concept of an increasing effective permittivity
with an increasing air gap. However, after some consideration, the case of the 3 electrode geometry was convincing enough to rely
on the values ascertained by the simulation software. 
\newline
In general, when it comes to evaluating the performance of a simulation the limited accuracy
of finite element methods is a concern. COMSOL gives the option to decrease the minimum size of each of the meshed elements.
A few different mesh sizes were used until the simulation converged such that it was guaranteed that the accuracy is sufficient. 

\section{Measures to improve the integrator}
The results indicate that the integrator introduces much noise at low frequencies. One might think that this does not influence the results as frequencies below 1kHz are not monitored, but this is not the case due to the spectral leakage of the FFT.  A high-pass filter at the output of the integrator could reduce these parasitic frequency components. This might reduce the measured swing at the output of the integrator and thus improve the results of the dielectric spectroscopy. 
Moreover, a more detailed analysis of the origin of these low-frequency components is necessary. It has already been shown that the external voltage source is not responsible for them, thus the integrator has to be investigated part by part in order to find the origin of this noise. 
It would also be interesting to completely evaluate the measured transfer function of the integrator. Since the integrator has a gain of 60 dB the input power has to be chosen very low in order to stay within the allowed input power range of the network analyzer. Finally, the input is so low that the integrator does add too much noise, what makes an identification of the transfer function impossible. For a proper evaluation of the transfer function an amplifier / attenuator is necessary. 

Another measure to reduce the offset actively is a feedback loop. A PI-controller in the feedback loop at the output can make sure that there an offset at the input is eliminated as described in \cite{thomas}. 

\section{Measures to improve the current transformer}
As shown in the chapter \ref{chp.results} the measurements with the current transformer deteriorates the confidence interval of the measured $tan(\delta)$ by orders of magnitude. This deterioration is reduced by the integrator, but it is still higher compared to the DLPCA. 
Furthermore, a more detailed analysis of the correlation between the NF of the current transformer (with and without integrator) for different frequencies and the confidence interval of the measured $tan(\delta)$ is necessary. This would provide detailed information on the impact of the quality of the current transformer on the precision of the dielectric spectroscopy. Thus, it is necessary to determine the origin of the noise. 


\section{Measures to improve the Debye model}
\label{debyemodel}
As shown in the results the measurements for the dielectric spectroscopy the theoretical $tan(\delta)$ is not within the confidence interval when measured with the DLPCA. This indicates that there are further parasitic impedances in the Debye model that were not considered. These might be prevented by evaluating the spectrum of the impedance of the Debye network. 


\section{Measurement and signal processing during dielectric spectroscopy}
The procedure of how the data is acquired described in \ref{spectroscopy}
could be improved as described below. The ADC has a maximum sampling rate of
1Mhz. As in the N-point DFT, the signals are implicitly assumed to be periodic,
the sinusoidal signals at different frequencies were sampled only along one period.
Using the formula for the frequency spacing of the FFT bins
\begin{equation}
 \Delta f=\frac{1}{\Delta t N}
\end{equation}
whereas N denotes the number of samples. 
Given the formula and the sampling algorithm, it can be deduced that the frequency spacing is
always exactly as large as the fundamental frequency of the recorded signal.
Consequently, only the signal frequency and its harmonics can be recorded appropriately.
Low frequency noise will be distributed among all of the bins and thus, decreases
the quality of the measurement in terms of SNR and NF. It is hard to predict how much of the noise will be added to the bin that should
monitor the signals amplitude. 
Another problem is that while a sinusoidal wave is in fact periodic, superimposed oscillations of other frequencies, such as the ones that
were recorded of the integrator in the setup, will cause a major difference between the first sample point and the last one. 
Given the FFT's properties this sudden jump that is caused by the assumption of the signal being periodic adds
disproportionally large amounts of high-frequency containments to the higher bins of the FFT. In this case, the SNR might be underestimated by an unknown factor. 
\newline
Furthermore, an increasing dependency of the size of the confidence intervals with respect to the noise figure of
the setup was determined. 
More experiments using a higher number of phase averages will certainly lower the variance in the experiment, however it would 
be highly advantageous to have a more specific relation between the size of the confidence intervals
and the noise figure, given the number of phase averages.
\begin{equation}
 95 \% CI=f\left(NF\right)
\end{equation}
Once this relation is established, one can define the maximum acceptable
interval size for a given number of phase averages and find the maximum noise figure, 
which in return gives a clear cut requirements to the hardware and software involved in the measurement setup.
For example, one would be able to tell by which factor the integrator has to improve the SNR in order to achieve
an adequate measurement outcome.


\section{Protective clamping capability of the current transformer}
The results prove that by simply adding 2 shunt TVS-Diodes the voltage can effectively be clamped to a value much lower
than the output voltage of the current transformer in case of sample breakdown.
One thing worth to note is that in the data sheet of diode, the clampling voltage is stated to be 12V. The measurements consistently show 17.5 V though.
This suggests that a simply soldering the diodes to ground means they are not operated as intended.
\newline
Another problem which potentially has more severe consequences is that the curves of the current through
the current transformer and the voltage generated by the device no longer overlap.
Those results suggest that a more sophisticated clamping mechanism needs to be investigated and tested.


