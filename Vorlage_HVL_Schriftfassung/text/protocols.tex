\chapter{Protocols}
\section{Protocol: Measurement of capacitance of two samples}
\large{Aim} \\
\begin{itemize}

\item Measurement of capacitances of two samples in order to derive the $\epsilon$ with a optically measured electrode distance 
\item Measurement of the deviation in capacitance due to slightly different insertion of the sample into the low-voltage cell. Measurement of the deviation in due to errors in the measurement devices. 
\item Optical measurement of electrode distance after curring the sample (done by Raphael F\"arber) 
\end{itemize}

\large{Used Instruments} 
\begin{itemize}
 \item Butterworth Filter: Alligator Technologies USBPGF-S1
 \item DLPCA: Femto DLPCA-2000
 \item Oscilloscope: LeCroy waveSurfer 24MXs-A
 \item External Voltage Supply:  XHR 20-50
 \item DAC: National Instruments cDAQ-9171

\end{itemize}


\large{Setup} \\
The input signal was  sinusoidal with 1kHz and $V_pp=16$. This signal is applied to two different samples in the low-voltage-setup. The current through the probe is measured with DLPCA, filtered with a Butterworth filter and then given to a ADC. 
Input signal and output current are measured and the division of these quantities allows to deduce $C*$.\\

Together with the optical measurement of d the effective $\epsilon$ can be calculated. 

Settings: \newline
Signal Generator:  sin, 1kHz, $V_pp=16V$
DLPCA:  automatically adapted gain
BW-Filter:  cutoff-frequency : 100kHz, Gain: 1

\large{Measurement} \\
25 measurements of the capacitance each time after the samples were taken out of the low-voltage cell and put in again. Each measurement comprises of 25 phase shifts. 

The measurements were conducted for two different samples.

Measured values for sample 2 is not usable as there is entrapped air near the electrode. This was realized when measuring the electrode distance after slicing of the sample.

\large{Surroundings} \\
Temperature: 
Humidity: 

\section{Protocol: Measurement of $\epsilon$ and tan($\delta$) of Debye Model}
\large{Aim} \\
\begin{itemize}
\item Measurement of the $tan(\delta)$ and $\epsilon$ for different frequencies beteween 1 kHz and 10kHz in order to compare the results with the theoretical values
\item  
\end{itemize}


\large{Used Instruments} 
\begin{itemize}
 \item Butterworth Filter: Alligator Technologies USBPGF-S1
 \item DLPCA: Femto DLPCA-2000
 \item Oscilloscope: LeCroy waveSurfer 24MXs-A
 \item External Voltage Supply:  XHR 20-50
 \item DAC: National Instruments cDAQ-9171
 \item Current Transformer: 

\end{itemize}


\large{Setup} \\
The input signal was  sinusoidal with 1kHz and $V_pp=15$. The signal is applied to the Debye-model in the low-voltage setup. The current through the Debye-model is firstly measured with DLPCA, filtered with a Butterworth filter and then given to a ADC. Secondly, just the current transformer is used to measure the current. Thirdly, the current transformer is used with an integrator. 
For all thre measurement setups the measurement is done with the complete Debye model and as a reference measurement just with $C_0$, i.e. $R_i$ and $C_i$ are removed. 

Settings: \newline
Signal Generator:  sin, 1kHz to10 kHz, $V_pp=16V$
DLPCA:  automatically adapted gain
BW-Filter:  cutoff-frequency : always double the input frequency, gain according to the following table

\large{Measurement} \\
Each time, 9 phase shifts were recorded. 

\large{Surroundings} \\
Temperature: 
Humidity: 


\section{Protocol: Measurement of Noise Figure and Noise Floor}
\large{Aim} \\
\begin{itemize}
\item Determination of the Noise figure with and without an integrator
\item Receiving noise flor 
\end{itemize}

\large{Surroundings} \\
Temperature:  
Humidity: 