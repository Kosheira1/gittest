\chapter{Protocols}
\section{Protocol: Measurement of capacitance of two samples}
This protocol documents the measurements for part 1. \\
{\Large Aim}
\begin{itemize}
\item Measurement of capacitances of two samples in order to derive the $\epsilon$ with a optically measured electrode distance 
\item Measurement of the deviation in capacitance due to slightly different insertion of the sample into the low-voltage cell. Measurement of the deviation in due to errors in the measurement devices. 
\item Optical measurement of electrode distance after curring the sample (done by Raphael F\"arber) 
\end{itemize}
 
\par
{\large Used Instruments}
\begin{itemize}
 \item Butterworth Filter: Alligator Technologies USBPGF-S1
 \item DLPCA: Femto DLPCA-2000
 \item Oscilloscope: LeCroy waveSurfer 24MXs-A
 \item External Voltage Supply:  XHR 20-50
 \item ADC: National Instruments cDAQ-9171

\end{itemize}


{\large Setup} 

The input signal was  sinusoidal with 1kHz and $V_{pp}=16V$. This signal is applied to two different samples in the low-voltage-setup. The current through the probe is amplified with the DLPCA, filtered with a Butterworth filter and then given to a ADC. 
Input signal and output current are measured and the division of these quantities allows to deduce $C^*$.\\
Together with the optical measurement of d the effective $\epsilon$ can be calculated. \\

{\large Settings:} \newline
Signal Generator:  sin, 1kHz, $V_pp=16V$\\
DLPCA:  automatically adapted gain\\
BW-Filter:  cutoff-frequency : 100kHz, Gain: 1

\large{Measurement} \\
25 phase averages were taken of the capacitance. The samples were taken out the low-voltage cell each time and put in again.

The measurements were conducted for two different samples.\\

Measured values for sample 2 are not usable as there is entrapped air near the electrode. This was realized when measuring the electrode distance after slicing the sample.

{\large Surroundings} \\
Temperature: 21.3 $^{\circ}$
Humidity: 

\section{Protocol: Measurement of $\epsilon$ and tan($\delta$) of the debye model}
{\large Aim} \\
\begin{itemize}
\item Measurement of the $tan(\delta)$ and $\epsilon$ for different frequencies beteween 1 kHz and 10kHz in order to compare the results with the theoretical values
\end{itemize}


\large{Used Instruments} 
\begin{itemize}
 \item Butterworth Filter: Alligator Technologies USBPGF-S1
 \item DLPCA: Femto DLPCA-2000
 \item Oscilloscope: LeCroy waveSurfer 24MXs-A
 \item External Voltage Supply:  XHR 20-50
 \item DAC: National Instruments cDAQ-9171
 \item Current Transformer: PEARSON CURRENT
MONITOR MODEL 2877

\end{itemize}


\large{Setup} \\
The input signal was  sinusoidal with frequencies . The signal is applied to the Debye-model in the low-voltage setup. The current through the Debye-model is firstly measured with DLPCA, filtered with a Butterworth filter and then given to a ADC. Secondly, just the current transformer is used to measure the current. Thirdly, the current transformer is used with an integrator. 
For all three setups the measurement is done with the complete Debye model first and as a reference measurement just with $C_0$, i.e. $R_i$ and $C_i$ are removed afterwards.\\ 

{\large Settings:} \newline
\begin{itemize}
\item Signal Generator:  sin, frequencies:500,1000,2000,2500,2600,2700,
2750,2800,2850,2900,2950,
3000,5000,10000,50000,\\
\item DLPCA:  automatically adapted gain
\item BW-Filter:  cutoff-frequency : always double the input frequency, gain according to the following table\\ coupling: without CT: DC, with CT: DC, with DC and Integrator: AC
\end{itemize}
\large{Measurement} \\
Each time, 9 phase averages were recorded. 

\large{Surroundings} \\
Temperature: 22.1
Humidity: 


\section{Protocol: Measurement of Noise Figure and Noise Floor}
\large{Aim} \\
\begin{itemize}
\item Determination of the Noise figure with and without an integrator 
\item Receiving noise flor 
\end{itemize}
\large{Used Instruments} 
\begin{itemize}
 \item Butterworth Filter: Alligator Technologies USBPGF-S1
 \item DLPCA: Femto DLPCA-2000
 \item Oscilloscope: LeCroy waveSurfer 24MXs-A
 \item External Voltage Supply:  XHR 20-50
 \item DAC: National Instruments cDAQ-9171
 \item Current Transformer: PEARSON CURRENT
MONITOR MODEL 2877

\end{itemize}


\large{Setup} \\
The input signal was  sinusoidal with frequencies . The signal is applied to the Debye-model in the low-voltage setup. The current through the Debye-model is firstly measured with DLPCA, filtered with a Butterworth filter and then given to a ADC. Secondly, just the current transformer is used to measure the current. Thirdly, the current transformer is used with an integrator. 
For all three setups the measurement is done with the complete Debye model first and as a reference measurement just with $C_0$, i.e. $R_i$ and $C_i$ are removed afterwards.\\ 

{\large Settings:} \newline
\begin{itemize}
\item Signal Generator:  sin,
frequencies: 500,1000,2000,2500,2600,2700,
2750,2800,2850,2900,2950, 3000,5000,10000,50000 Hz\\
\item DLPCA:  automatically adapted gain
\item BW-Filter:  cutoff-frequency : always double the input frequency, gain according to the following table\\ coupling: without CT: DC, with CT: DC, with DC and Integrator: AC
\end{itemize}
{\large Measurement} \\
Each time, 9 phase averages were recorded. 

\large{Surroundings} \\
Temperature:  
Humidity: 