\chapter{Protocols}
\section{Protocol: Measurement of capacitance of two samples}
\large{Aim} \\
\begin{itemize}

\item Measurement of capacitances of two samples in order to derive the $\epsilon$ with a optically measured electrode distance 
\item Measurement of the deviation in capacitance due to slightly different insertion of the sample into the low-voltage cell. Measurement of the deviation in due to errors in the measurement devices. 
\item Optical measurement of electrode distance after curring the sample (done by Raphael F\"arber) 
\end{itemize}

\large{Used Instruments} 
\begin{itemize}
 \item Butterworth Filter: Alligator Technologies 
 \item DLPCA: 
 \item Oscilloscope: LeCroy
 \item External Voltage Supply: 
 \item DAC: National Instruments 

\end{itemize}


\large{Setup} \\
The input signal was  sinusoidal with 1kHz and $V_pp=16$. This signal is applied to two different samples in the low-voltage-setup. The current through the probe is measured with DLPCA, filtered with a Butterworth filter and then given to a ADC. 
Input signal and output current are measured and the division of these quantities allows to deduce $C*$.\\

Together with 

Settings: \newline
Signal Generator:  sin, 1kHz, $V_pp=16$
DLPCA:  automatically adapted gain
BW-Filter:  cutoff-frequency : 100kHz, Gain: 1

\large{Measurement} \\
25 measurements of the capacitance each time after the samples were taken out of the low-voltage cell and put in again. Each measurement comprises of 25 phase shifts. 

All measurements were done for two samples. 



\section{Protocol: Measurement of epsilon and tan(delta) of Debye Model}
\large{Aim} \\
\begin{itemize}
\item Measurement of the $tan(\delta)$ and $\epsilon$ for different frequencies beteween 1 kHz and 10kHz in order to compare the results with the theoretical values
\item  
\end{itemize}



\section{Protocol: Measurement of Noise Figure and Noise Floor}
\large{Aim} \\
\begin{itemize}
\item Determination of the Noise figure with and without an integrator
\item  
\end{itemize}

