\chapter{Methods}
\section{Part 1}
\subsection{Simulation of vacuum capacitance} 
The dependency of the vacuum capacity $C_0$ has to be simulated numerically for the high-voltage setup as well as for the low-voltage setup. 
\begin{figure}[htbp]
	\centering
	\includegraphics{figures/COMSOL_Beispielbild.jpg}		
	\caption[Kurze Abbildungsbeschreibung]{Field simulation of the low-voltage test cell (0.5 mm electrode distance)} \ref{sec.analysecurrent}
	\label{fig.waveforms}
\end{figure}
 
There are several systematic errors that influence the vacuum capacity. One factor is the air gap between the wall of the low-voltage setup and the specimen. This air gap is unavoidable in order to be able to remove the specimen from the pouring setup. This air gap should theoretically be 0.5 mm, but it is expected to vary in worst-case between 0 mm and 1 mm. The lookup table of d based on $C_0$ is based on an air gap of 0.5 mm, therefore the deviation from this value has to be investigated. 
Another systematic error is the deviation of the height of the specimen. During  the pouring process the height of the specimen cannot be . The maximum deviation is expected to be $\pm$ 0.5 mm. This deviation might occur during the production process of each new specimen. Thus, if the electrode distance of a specimen should be estimated by the method described in !!! this 
A third influence is the 

\section{Part 2}
\section{Signal analysis of the Debye model}

In order to emulate different dielectrica with their own respective dielectric loss tangent, different combinations of circuit elements were used. Each different circuit accounts for another loss tangent with respect to frequency. With the objective to assess the performance of the current transformer, reasonably high values for the $tan\left(\delta\right)$ were assumed (i.e. 0.05 to 0.2) since a lower loss tangent would require a higher resolution on the part of the current measurement.


\begin{figure}[htbp]
	\centering
	\includegraphics{figures/COMSOL_Beispielbild.jpg}		
	\caption[Kurze Abbildungsbeschreibung]{Electron drift currents in Ar at 30 Td and in CO$_2$ at 65 Td, the latter was divided by 10 and shifted by 0.2 $\mu$s. Dotted lines are averages of measured waveforms, solid lines are fits of Eq. XX. $T$ marks the electron transit time, and the markers $T_1$ to $T_3$ are explained in section.} \ref{sec.COMSOL_Beispielbild}
	\label{fig.waveforms}
\end{figure}


\subsection{Design of a holder for the Debye Equivalent Circuit}
For a lossy medium the Debye Network consists at least of a vacuum capacity $C_0$ and the term for the charging of the additional capacitance. There might be . Thus, it is reasonable to design a holder for the network that can hold three strands with two components each. Moreover, it has to fit into the low-voltage and high-voltage test cell. 

\begin{figure}
\includegraphics[width=\textwidth]{figures/Method/CAD_MODEL/Gesamtanordnung.jpg}
\end{figure}
\newpage

\begin{sidewaysfigure}
\includegraphics[width=0.99\textwidth]{figures/Gesamtanordnung.pdf}
    \caption{Property profile of the diverse library compared to the compound pool.}
    
   \end{sidewaysfigure}	
\newpage
    

\subsection{Measurement of the textepsilon and the tan(textdelta) for the Debye model}

The aim is to have a current flwoing though the Debye model in the low-voltage setup that is oft the same order as a current in the high voltage-setup. In order to have a comparable current for a voltage that is $1/1000$ of the high-voltage setup a capacitance $C_0$ for the Debye is selected 1000 times higher than the Capacitance of the samples. As the sample has a capacitance of 3.44 pF its equivalent $C_0$ in the debye-model is 3.3 nF. In order to get a tan($\delta$) of !!! $R_i$ is chosen as and $C_i$ is chosen as .
\begin{figure}
	\includegraphics{figures/Method/debye-modell.jpg}	
	\caption{Debye model for tan($tan(\delta)$)= }	
\end{figure}






\subsection{Integrator}
\subsection{Advantages of using an integrator}
As the used 16-bit DAC has a limited resolution it is advantageous to use an integrator in order to spread the frequencies of the signal over a larger timescale. This makes it possible to detect the different frequencies more accurately with the same DAC. 
A second reason to make use of an integrator is due to the fact that it adds an additional factor of $1/f$ to the Fourier spectrum of the output voltage signal. As the Fourier spectrum of the input signal is as well characterized by a $1/f$ decrease the integrator transforms the pulse signal back to a trapezoidal signal with a $1/f$-dependency in the Fourier spectrum. Since a $1/f$ dependency in the output signal is preferable to a constant Fourier spectrum due to better resolution for the first harmonic, the integrator improves the measurement as well. 
\subsection{Design of integrator}
The integrator should have two functions. On the hand side it should integrate signal on the other hand it should amplify the signal by a factor of 1000. This is due to the fact small below-PD-currents ($\sim$mA) should be measured. With a CT-sensitivity of $1V/A$ an amplification of 1000 is reasonable to use the range of measurable inputs of $\pm$ 10 V. These requirements can be achieved by a non-inverting integrator. 

\begin{sidewaysfigure}
\includegraphics[width=0.99\textwidth]{figures/Method/integrator/PCB_Integrator.png}
    \caption{Property profile of the diverse library compared to the compound pool.}
    
    \end{sidewaysfigure}	
    
    
    \newpage
    
    	\begin{sidewaysfigure}
\includegraphics[width=0.99\textwidth]{figures/Method/integrator/schematic.jpg}
 \caption{Property profile of the diverse library compared to the compound pool.}
  \end{sidewaysfigure}	

The integrator has the following theoretical transfer function.

\begin{figure}

\includegraphics[width=\textwidth]{figures/Method/integrator/transferfunction_int.jpg}





\caption[Kurze Abbildungsbeschreibung]{Bode plot of an integrator } \ref{sec.Bodeplot}
\end{figure}
	



\subsection{Protective Clamping Capability of the current transformer}

